\documentclass[aspectratio=169,xcolor=dvipsnames]{beamer}
\usepackage[T2A]{fontenc}
\usepackage[utf8]{inputenc}
\usepackage[english,russian]{babel}
\usepackage{epsfig}
\usepackage{verbatim}
\usepackage{gnuplottex}
\usepackage{listings}
\usepackage{hyperref}
\usepackage{graphicx}
\usepackage{booktabs}
 

\usetheme{Simple}
%\setbeameroption{show notes}

\subtitle[короткое название]{Петрозаводский государственный университет
Институт математики и информационных технологий
Кафедра Информатики и математического обеспечения}

\author[Автор]{Мельников Илья Евгеньевич 22207}
\institute[ПетрГУ]
{
\Large{\textcolor{blue}{Разработка 2D Roguelike игры на движке Unity}} \\\vspace{20px}
\small{Направление подготовки бакалавриата
09.03.04 Программная инженерия
Профиль направления подготовки бакалавриата
“Системное и прикладное программное обеспечение”} \vspace{20px}

\normalsize{Научный руководитель: к.т.н., доцент кафедры ИМО, C. А. Марченков}

    \vskip 3pt
}
\date{} 

\begin{document}

\begin{frame}
    \titlepage
\end{frame}

\begin{frame}{Выбор жанра и игрового движка}
\begin{figure}
\includegraphics<1->[width=1\linewidth]{pictures/1.jpg}
\end{figure}
\end{frame}

\begin{frame}{Цели и задачи}
\begin{center}
\textbf{Цели}
\end{center}
\begin{itemize}
    \item Приобрести навыки и опыт работы с движком Unity.
    \item Повысить свою квалификацию по ходу работы с C\# и системой Tilemap для Unity.
    \item Закрепить имеющиеся навыки во время работы с:
\end{itemize}

\begin{center}
\textbf{Задачи}
\end{center}
\begin{itemize}
    \item Освоиться с интерфейсом Unity.
    \item Создание Roguelike игры.
    \item Сборка готового билда для скачивания другими пользователями.
    \item Добавление контента в игру.
\end{itemize}
\end{frame}

\begin{frame}{Создание проекта}
\begin{figure}
\includegraphics<1->[width=0.9\linewidth]{pictures/project.png}
\end{figure}
\end{frame}

\begin{frame}{Подготовка}
\begin{figure}
\includegraphics<1->[width=0.2\linewidth]{pictures/folders.png}
\includegraphics<1->[width=0.7\linewidth]{pictures/layout.png}
\end{figure}
\end{frame}

\begin{frame}{Выбор спрайтов}
\begin{figure}
\includegraphics<1->[width=0.4\linewidth]{pictures/Atlas.png}
\includegraphics<1->[width=0.2\linewidth]{pictures/sword.png}
\includegraphics<1->[width=0.2\linewidth]{pictures/guy.png}
\end{figure}
\end{frame}

\begin{frame}{Написание кода}
\begin{figure}

\includegraphics<1->[width=0.35\linewidth]{pictures/player1.png}
\includegraphics<1->[width=0.4\linewidth]{pictures/boxcoll.png}

Player.cs \ \ \ \ \ \ \ \ \ \ \ \ \ \ \ \ \ \ \ \ \ \ \ \ \ \ \ \ \ \ \ \ \ \ \ \ Box Collider
\end{figure}
\end{frame}


\begin{frame}{Написание кода}
\begin{figure}

\includegraphics<1->[width=0.35\linewidth]{pictures/level1.png}
\includegraphics<1->[width=0.4\linewidth]{pictures/level2.png}

Level 1. Main \ \ \ \ \ \ \ \ \ \ \ \ \ \ \ \ \ \ \ \ \ \ \ \ Level 2. Dungeon
\end{figure}
\end{frame}

\begin{frame}{Написание кода}
\begin{figure}

\includegraphics<1->[width=0.4\linewidth]{pictures/coll.png}
\includegraphics<1->[width=0.4\linewidth]{pictures/coll1.png}

Collidable.cs \ \ \ \ \ \ \ \ \ \ \ \ \ \ \ \ \ \ \ \ \ \ \ \ \ \ \ \ \ \ \ Collectable.cs
\end{figure}
\end{frame}

\begin{frame}{Написание кода}
\begin{figure}

\includegraphics<1->[width=0.4\linewidth]{pictures/chest.png}
\includegraphics<1->[width=0.4\linewidth]{pictures/portal.png}

Chest.cs \ \ \ \ \ \ \ \ \ \ \ \ \ \ \ \ \ \ \ \ \ \ \ \ \ \ \ \ \ \ \ \ \ \ \ \ \ \ \ Portal.cs
\end{figure}
\end{frame}

\begin{frame}{Написание кода}
\begin{figure}

\includegraphics<1->[scale=0.24]{pictures/gm.png}
\includegraphics<1->[scale=0.35]{pictures/fm.png}

GameManager.cs \ \ \ \ \ \ \ \ \ \ \ \ \ \ \ \ \ \ \ \ \ FloatingTextManager.cs
\end{figure}
\end{frame}

\begin{frame}{Боевая система}
\begin{figure}

\includegraphics<1->[width=0.4\linewidth]{pictures/anim.png}
\includegraphics<1->[width=0.4\linewidth]{pictures/mover.png}

Анимация \ \ \ \ \ \ \ \ \ \ \ \ \ \ \ \ \ \ \ \ \ \ \ \ \ \ \ \ \ \ \ \ \ \ \ \  Mover.cs
\end{figure}
\end{frame}

\begin{frame}{Анимация}
\ \\
\begin{figure}
\includegraphics<1->[scale=0.4]{pictures/anim1.png}

Раздел анимации \\
\
\\
\
\\
\includegraphics<1->[scale=0.4]{pictures/anim2.png} 

Раздел аниматора
\end{figure}
\end{frame}

\begin{frame}{Меню}
\begin{figure}
\includegraphics<1->[scale=0.4]{pictures/menu.png}
\includegraphics<1->[scale=1.4]{pictures/hud.png}

\ \ \ \ \ \ \ \ \ \ \ \ \ \ \ \ \ \ \ \ Меню \ \ \ \ \ \ \ \ \ \ \ \ \ \ \ \ \ \ \ \ \ \ \ \ \ \ \ \ \ \ \ \ \ \ \ \ \ \ \ \ Интерфейс
\end{figure}
\end{frame}

\begin{frame}{Добавление контента}
\begin{figure}
\includegraphics<1->[scale=0.35]{pictures/boss.png}
\includegraphics<1->[scale=1.05]{pictures/prefab.png}

\ \ \ Пример уровня \ \ \ \ \ \ \ \ \ \ \ \ \ \ \ \ \ \ \ \ \ \ \ \ \ \ \ \ \ \ \ \ \ \ \ \ \ \ \ \ \ \ Prefab
\end{figure}
\end{frame}

\begin{frame}{Заключение}

\begin{center}
\textbf{Выполнены задачи}
\end{center}
\begin{itemize}
    \item Ознакомление с игровым движком Unity.
    \item Создана Roguelike игра.
    \item Собран готовый билд для скачивания желающими.
    \item Добавлен контент в виде законченного уровня.
\end{itemize}

\vfill

URL: https://github.com/godofredovascancelos/MyDungeon1
\end{frame}

\end{document}